\section{Justificaciones}
\subsection{Etiquetado empleado}
Para el etiquetado de las tomas en cada planta hemos decidido utilizar la siguiente nomenclatura:
\begin{center}
    \textbf{RX-ZY-N}
\end{center}

\begin{itemize}
    \item \textbf{RX:} Indica el rack al que estará conectada la toma en cuestión, puediendo ser \textbf{0} si se refiere al rack de la planta baja o \textbf{1} si es al rack de la primera planta.
    \item \textbf{ZY:} Indica la zona en la que se encuentra la toma. Cada planta esta dividida en 3 zonas: \textbf{N (Norte)}, \textbf{E (Este)} y \textbf{O (Oeste)}.
    \item \textbf{N:} Indica el número asignado a dicha toma.
\end{itemize}

En el etiquetado de los switches se ha seguido la siguiente nomenclatura:

\begin{center}
    \textbf{RX-SY-0/1}
\end{center}

\begin{itemize}
    \item \textbf{RX:} Indica el rack al que estará conectada la toma en cuestión, pudiendo ser \textbf{0} si se refiere al rack de la planta baja o \textbf{1} si es al rack de la primera planta.
    \item \textbf{SY:} Indica el switch al que esta conectada la toma en el rack. La Y indica si se trata de una de las 3 zonas anteriormente mencionadas: \textbf{N (Norte)}, \textbf{E (Este)} y \textbf{O (Oeste)}.
\end{itemize}

\subsection{Medio físico del cableado horizontal}
Para empezar, en la \textbf{planta baja} (R0) hemos habilitado la sala del Rack en el almacén de la zona nordestes de la planta debido a que es la ubicación más eficiente para ello. Esta sala es lo suficientemente grande como para albergar este equipo y de hecho, también es buena una correcta ventilación. La opción idónea hubiera sido colocarlo lo más cercano posible al patio central, pero no tenemos ninguna localización segura para ello. Es por eso que hemos decidido esta ubicación, ahorrándonos así cualquier tipo de obra. \\ Desde aquí discurrirían todo el cableado a través de falso techo por toda la planta, bajando hacia los equipos en las áreas de trabajo.\\

Por otro lado, el  rack en la \textbf{primera planta} (R1) estaría ubicado en el SITE de Informática, ya que es una sala ya acondicionada para los equipos de este tipo. En esta ya tenemos ventilación acorde al propósito y desde aquí, podemos extender toda la red cableada, de nuevo, a través de falso techo y bajando hacia las áreas de trabajo.\\

Por último, es necesario aclarar en cuanto a los puntos de acceso para \textbf{la señal Wi-Fi} que los hemos tenido cuanto pero no los hemos ubicado en el mapa. Esta decisión viene debida a que nos faltaría información por parte de los clientes del edificio, con los que tendríamos que conversar sobre cuales serían las zonas más concurridas, y donde ellos tienen previsto que sean puntos de reunión y concentración, como por ejemplo, el comedor.\\ Lo mismo ocurriría con las \textbf{cámaras de seguridad}; cuales serían las zonas prioritarios de vigilancia y cual sería la ubicación de estos dispositivos sería un tema a comentar con los clientes. Todo el cableado horizontal estaría preparado aun así para cualquier instalación de cámaras y Wi-Fi.
\subsection{Medio físico cableado vertical}
El orden de los dispositivos del rack 2 se ajusta al camino lógico que siguen los datos, como una estructura en cascada. En la parte superior de ambos racks se ha colocado el panel de fibra óptica (previene la acumulación de polvo), justo debajo tenemos los switches organizados por zonas (en la imagen tenemos una posibilidad) y abajo del todo los paneles de conexión, primero el de Internet y al final el de conexiones telefónicas. El router queda ubicado en el rack de edificio, en la planta baja, así como la centralita. En el cableado vertical de datos, para evitar cuellos de botella, utilizaremos fibra multimodo, ya que la distancia que tiene que recorrer el cable es corta y es mucho más tolerante al desalineamiento que la monomodo. Respecto al de voz, se utilizarán mangueras de cable multipar, siguiendo la normativa. De cara al futuro, el cableado estaría preparado para cualquier necesidad de aumento del ancho de banda.
\subsection{¿Porqué sólo 2 Racks?}
Para este plano de edificio con dos plantas sólo hemos empleado 2 Racks, no es tan grande la extensión de las plantas para tener que utilizar más de  un armario por planta.
\subsection{Número y ubicación de los distribuidores}
Utilizaremos un solo armario por planta (siendo el de la planta baja el rack de edificio). Hemos decidido ubicarlo en la habitación condicionada en la zona Norte, ya que esta zona está centrada (lo que reduce las longitudes de los cables) y cercana al patio (lo que ayuda en el acondicionamiento). En la planta alta el distribuidor estará justo arriba del distribuidor de la planta baja, esto permitirá que el cableado vertical sea lo más corto posible.