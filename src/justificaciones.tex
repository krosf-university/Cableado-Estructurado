\section{Justificaciones}
\begin{itemize}
    \item \textbf{Etiquetado empleado: }\\ Para el etiquetado de las tomas en cada planta hemos decidido utilizar la siguiente nomenclatura: \centerline{\textbf{RX-ZY-N}}
    \begin{itemize}
        \item RX: Indica el rack al que estará conectada la toma en cuestión, puediendo ser 0 si se refiere al rack de la planta baja o 1 si es al rack de la primera planta.
        \item ZY: Indica la zona en la que se encuentra la toma. Cada planta esta dividida en 3 zonas: N (Norte), E (Este) y O (Oeste).
        \item N: Indica el número asignado a dicha toma.
    \end{itemize}
    \vspace{0.5cm}
    En el etiquetado de los switches se ha seguido la siguiente nomenclatura: \centerline{\textbf{RX-SY-0/1}}
    \begin{itemize}
        \item RX: Indica el rack al que estará conectada la toma en cuestión, puediendo ser 0 si se refiere al rack de la planta baja o 1 si es al rack de la primera planta.
        \item SY: Indica el switch al que esta conectada la toma en el rack. La Y indica si se trata de una de las 3 zonas anteriormente mencionadas: E, N u O.
        \item \textbf{ACLARACIÓN:} Indicamos los pares e impares en los etiquetados de forma que, cuando se trata de un puesto de trabajo, el número par es de voz (RJ-11) y el impar se trata de ethernet (RJ-45)- \\ Además esto sive para cuando son puestos de trabajo, cuando estos se acaban siguen otro dispositivios y ya no sigue la misma norma.
    \end{itemize}
    \vspace{1cm}
    \item \textbf{Medio físico del área de trabajo}
    \vspace{1cm}
    \item \textbf{Medio físico del cableado horizontal}
    \vspace{1cm}
    \item \textbf{Medio físico del cableado vertical}
    \vspace{1cm}
    \item \textbf{¿Porqué sólo 2 Racks?}\\ Para este plano de edificio con dos plantas sólo hemos empleado 2 Racks
\end{itemize}